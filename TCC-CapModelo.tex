% CAP�TULO MODELO
\chapter{Primeiro cap�tulo}
\label{chp:PrimeiroCapitulo}

Texto qualquer\ldots

Um exemplo de cita��o \cite{Furaste:2007}.

Um exemplo de formata��o usando \textbf{negrito} e \emph{it�lico}.

Um exemplo de nota de rodap�, por exemplo, GPL\footnote{General Public License}.

Criando uma lista de itens:

\begin{itemize}
	\item Item 1;
	\item Item 2;
	\item Item 3.
\end{itemize}

Criando uma lista de itens numerados:

\begin{enumerate}
	\item Primeiro item;
	\item Segundo item;
	\item Terceiro item.
\end{enumerate}

\chapter{Segundo cap�tulo}

Texto qualquer\ldots

\section{Primeira se��o}

Texto qualquer\ldots

Inserindo uma imagem.

\begin{figure}[htp]
	\centering
	%\includegraphics[width=.9\textwidth]{imgs/BrasaoCaxias}
	\includegraphics[scale=0.5]{imgs/BrasaoCaxias}
	\caption{Bras�o de Caxias do Sul}
	Fonte: \cite{Furaste:2007}
	\label{fgr:BrasaoCaxias}
\end{figure}

\chapter{Terceiro cap�tulo}

Texto qualquer\ldots

Inserindo uma tabela.

\begin{table}[ht]
	\centering
	\begin{tabular}{c|c}
	\hline
	Caracter & Seq��ncia de Escape\\
	\hline\hline
	\&  & \&amp;\\
	$<$ & \&lt;\\
	$>$ & \&gt;\\
	"   & \&quot;\\
	'   & \&\#39;\\
	\hline
	\end{tabular}
	\caption{Caracteres Especiais no XML}
	Fonte: \cite{Furaste:2007}
	\label{tbl:CaracteresEspeciais}
\end{table}

\section{Segunda se��o}

Texto qualquer\ldots

\subsection{Primeira subse��o}

Texto qualquer\ldots

Exemplo de refer�ncia para cap�tulo \ref{chp:PrimeiroCapitulo}.

Exemplo de refer�ncia para imagem \ref{fgr:BrasaoCaxias}.

Exemplo de refer�ncia para tabela \ref{tbl:CaracteresEspeciais}.

Exemplo de refer�ncia para c�digo fonte \ref{lst:Basico}.

\chapter{Quarto cap�tulo}

Texto qualquer\ldots

Inserindo c�digo fonte.

\singlespace % Seta espa�amento simples para o c�digo
\lstset{language=c++} % Linguagem do c�digo
\lstinputlisting[caption=Programa B�sico em C++, label=lst:Basico]{codes/Basico.cpp}
\onehalfspace % Seta espa�amento 1,5 novamente