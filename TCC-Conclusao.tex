% CONCLUS�O
\chapter{Conclus�o}

Durante o desenvolvimento desse trabalho foram realizados uma s�rie de estudos sobre bioinform�tica, biologia molecular, bancos de dados biol�gicos, biologia de sistemas, ontologia g�nica, entre outros, com o objetivo de entender os conceitos envolvidos e o fluxo de pesquisa de uma doen�a gen�tica, executado pelo especialista e/ou bi�logo. Tamb�m foram desenvolvidos artefatos visando � implementa��o do sistema e, ap�s o sistema estar implementado, foram descritas as modifica��es no projeto, problemas enfrentados e o resultado final. Por fim, foram realizados quatro estudos de caso com mestres e doutores das �reas da biologia e da inform�tica, no qual pode ser feita uma avalia��o do sistema.

Como contribui��o desse trabalho, foi desenvolvido um prot�tipo de sistema que automatiza e documenta o fluxo de pesquisa de uma doen�a g�nica, integrando os dados dos \emph{sites} do OMIM e do STRING e com isso simplificando o trabalho dos especialistas e/ou bi�logos.

Mesmo o prot�tipo tendo alguns problemas, documentados nesse trabalho, o sistema teve uma boa avalia��o e, principalmente, os especialista da �rea da biologia o consideraram �til e sugeriram uma s�rie de possibilidades de continua��o para esse trabalho.

Para trabalhos futuros, o sistema pode ser melhorado para atender melhor os seus usu�rios, por exemplo, adicionando funcionalidades como o refinamento das pesquisas com filtros por arquivos XML das redes, imagens das redes, entre outros, ou possibilitar o compartilhamento das pesquisas entre os usu�rios. E, visto que, esse trabalho mostrou aos bi�logos que � poss�vel integrar dados biol�gicos e automatizar fluxos de pesquisa, acredito que agora eles tenham in�meras sugest�es de trabalhos futuros interessantes.

O sistema \emph{web} BioNet, desenvolvido nesse trabalho, est� dispon�vel para ser utilizado e testado em: $<$\url{http://www.biosoft.bio.br}$>$